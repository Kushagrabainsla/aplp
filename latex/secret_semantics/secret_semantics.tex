\documentclass[11pt]{article}

\usepackage{amsmath}
\usepackage{amssymb}
\usepackage{mathpartir}
\usepackage{geometry}
\usepackage{enumitem}

\geometry{margin=1in}

\title{Big-Step Operational Semantics with Security Labels}
\author{}
\date{}

\begin{document}
\maketitle

\section{Overview}

This document defines a big-step operational semantics for a small expression
language extended with a \texttt{secret} construct.
The goal is to track whether values are \emph{public} or \emph{secret}, and to
ensure that secret information cannot be leaked, even through control flow.

The semantics enforces \emph{non-interference}: secret data may influence
computation, but it may never influence publicly observable results.

\section{Syntax}

The expressions of the language are:
\[
\begin{array}{rcl}
e &::=& \texttt{true} \mid \texttt{false} \mid i \\
  &\mid& \texttt{if } e \texttt{ then } e \texttt{ else } e \\
  &\mid& \texttt{succ } e \\
  &\mid& \texttt{pred } e \\
  &\mid& \texttt{secret } e
\end{array}
\]
where $i \in \mathbb{Z}$.

\section{Values and Security Labels}

The set of values is:
\[
v ::= \texttt{true} \mid \texttt{false} \mid i
\]

Each value is paired with a \emph{security label}:
\[
\ell ::= L \mid H
\]

\begin{itemize}[leftmargin=*]
  \item $L$ (low) means public information
  \item $H$ (high) means secret information
\end{itemize}

An evaluated expression produces a labeled value of the form $v^\ell$.

\section{Evaluation Judgement}

The big-step evaluation relation is written as:
\[
e \Downarrow v^\ell
\]

This should be read as:

\begin{quote}
Expression $e$ evaluates to value $v$ with security label $\ell$.
\end{quote}

\section{Operational Semantics}

\subsection{Base Values}

Constants are always public, since they do not depend on any secret input.

\begin{mathpar}
\inferrule
{ }
{\texttt{true} \Downarrow \texttt{true}^L}

\inferrule
{ }
{\texttt{false} \Downarrow \texttt{false}^L}

\inferrule
{ }
{i \Downarrow i^L}
\end{mathpar}

\subsection{Secret}

The \texttt{secret} construct explicitly marks data as private.
It evaluates its subexpression and upgrades the label to $H$.

\begin{mathpar}
\inferrule
{ e \Downarrow v^\ell }
{\texttt{secret } e \Downarrow v^H}
\end{mathpar}

\paragraph{Intuition.}
The value itself is unchanged, but from this point onward it is treated as
secret and cannot be safely observed.

\subsection{Arithmetic Operations}

Arithmetic operations propagate the security label of their argument.

\begin{mathpar}
\inferrule
{ e \Downarrow i^\ell }
{\texttt{succ } e \Downarrow (i+1)^\ell}

\inferrule
{ e \Downarrow i^\ell }
{\texttt{pred } e \Downarrow (i-1)^\ell}
\end{mathpar}

\paragraph{Intuition.}
If an arithmetic result depends on secret data, then the result must also be
secret. This captures \emph{explicit information flow}.

\subsection{Conditionals}

Conditionals are the main source of \emph{implicit information flow}.
To prevent leaks, the label of the result must depend on both:
\begin{itemize}[leftmargin=*]
  \item the condition
  \item the chosen branch
\end{itemize}

We define the label join operator $\sqcup$ as:
\[
\begin{array}{c|cc}
\sqcup & L & H \\
\hline
L & L & H \\
H & H & H
\end{array}
\]

\begin{mathpar}
\inferrule
{ e_1 \Downarrow \texttt{true}^{\ell_1} \\ e_2 \Downarrow v^{\ell_2} }
{\texttt{if } e_1 \texttt{ then } e_2 \texttt{ else } e_3
 \Downarrow v^{\ell_1 \sqcup \ell_2}}

\inferrule
{ e_1 \Downarrow \texttt{false}^{\ell_1} \\ e_3 \Downarrow v^{\ell_3} }
{\texttt{if } e_1 \texttt{ then } e_2 \texttt{ else } e_3
 \Downarrow v^{\ell_1 \sqcup \ell_3}}
\end{mathpar}

\paragraph{Key idea.}
If the condition is secret, then the observer must not be able to tell which
branch was taken. Therefore, the result is labeled $H$ even if the branch
itself produces a public value.

\section{Security Guarantee}

This semantics enforces \emph{non-interference}:

\begin{itemize}[leftmargin=*]
  \item Secret data may influence computation
  \item Any value influenced by secret data is labeled $H$
  \item There is no rule that converts $H$ back to $L$
\end{itemize}

As a result, attackers cannot write code that leaks secret information,
even indirectly through control flow.

\section{Example}

\[
\texttt{succ (secret 3)} \Downarrow 4^H
\qquad
\texttt{succ 3} \Downarrow 4^L
\]

\noindent
The difference in labels reflects whether the computation depends on secret
information.

\end{document}
